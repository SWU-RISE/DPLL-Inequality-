\documentclass{llncs}
\usepackage{comment}
\usepackage{graphicx}
\usepackage{amsmath}
\usepackage{indentfirst}
\usepackage{amsfonts}
\usepackage{latexsym,amsmath,epsfig,amssymb,graphics}
\usepackage{color}
\usepackage{listings}
\usepackage{stmaryrd}
\usepackage{caption}
\usepackage[ruled]{algorithm2e}

\usepackage[english]{babel}
\newtheorem{conj}{Conjecture}

%自定义的宏

\newcommand\numberthis{\addtocounter{equation}{1}\tag{\theequation}}

\newcommand{\vv}{\mathbf{V}}
\newcommand{\ee}{\mathbf{E}}
\newcommand{\GG}{\mathbf{G}}


\newcommand{\yy}{\mathbf{y}}
\newcommand{\xx}{\mathbf{x}}
\newcommand{\ZZ}{\mathbb{Z}}
\newcommand{\QQ}{\mathcal{Q}}
\newcommand{\CC}{\mathbb{C}}
\newcommand{\RR}{\mathbb{R}}
\newcommand{\PP}{\mathcal{P}}
\newcommand{\NN}{\mathbb{N}}

\newcommand{\PE}{\mathbf{P}}
\def \HB {{\mathcal{B}}}
\def \KK {{\mathcal{K}}}
\def \SS {{\mathcal{S}}}
\def \BB {{\mathbf{B}}}
\def \CC {{\mathbf{C}}}
\def \DD {{\mathbf{D}}}
\def \HH {{\mathbf{H}}}
\def \FF {{\mathbf{F}}}
\def \GG {{\mathbf{G}}}
\def \II {{\mathbf{I}}}
\def \zero  {{\mathbf{0}}}

\def \rhs {{\mathbf{b}}}

\def \bb {{\mathbf{\beta}}}
\newcommand{\cc}{\mathbf{c}}
\def \uu {{\mathbf{\mu}}}

\def \ll {{\mathbf{\lambda}}}
\def \ww {{\mathbf{w}}}

\def \ksi {{\mathbf{\xi}}}

\def \LIDER  {{\mathcal L}}
%\def \II  {{\vec{I}}}

\def \UU {{\mathbf{U}}}
\def \LL {{\mathbf{L}}}

\def \RE  {{\vec{R}}}

\def \SOFTD {{\textit {soft demand}}}
\def \HARDD{{\textit{hard demand}}}
\def  \MCF {{{\bf MCF}}}
\def  \mMCF {{{\bf min-MCF}}}
\def \MMCF  {{{\bf MMCF}}}
\def \MIN {{{\bf \min}}}
\def  \AA {{\mathbf{A}}}

\newtheorem{mydef}{Definition}


\DeclareMathOperator*{\argmin}{arg\,min}
% 图片路径
\graphicspath{{img/}}

\title{DPLL(Inequality) for Solving Polynomial Constrained Problems}
\author{Liyun Dai}
\institute{RISE, Southwest University, Chongqing, China \\
	\email{dailiyun@swu.edu.cn}
}


\begin{document}
	
	\maketitle
%	
%\input{section/intro}
%
%\begin{definition}[Rule1]
%	Given a polynomial $f(\xx)\in \ZZ[\xx]$ and $  f=\sum_{i=1}^s\delta_i(\sum_{j=1}^{n_i}a_{i,j}\xx^{k_{i,j}})^2$, where $\delta_i,a_{i,j}\in \RR, k_{i,j}\in \NN^{|\xx|}$. We use $\pi_1\left({\{\delta_i,a_{i,j} \}}\right)$ denote the number of  irrational elements which occur  in set $\{\delta_i,a_{i,j} \}$. We say SOS decomposition $f=\sum_{i=1}^s\delta_i(\sum_{j=1}^{n_i}a_{i,j}\xx^{k_{i,j}})^2$ is more beautiful than decomposition $f=\sum_{i=1}^{s'}\delta'_i(\sum_{j=1}^{n'_i}a'_{i,j}\xx^{k'_{i,j}})^2$ if $\pi_1\left({\{\delta_i,a_{i,j} \}}\right)\le \pi_1\left({\{\delta'_i,a'_{i,j} \}}\right).$
%	
%\end{definition}
%
%\begin{example}
%Let $f=x^2+4x+7$, then SOS 	decomposition $f=(x+2)^2+3$ is more beautiful than $f=\frac{1}{2}(x+\sqrt{2})^2+\frac{1}{2}(x+4-\sqrt{2})^2+4\sqrt{2}-3$.
%\end{example}
%
%
%\begin{definition}[Rule2]
%	Given a polynomial $f(\xx)\in \ZZ[\xx]$ and $  f=\sum_{i=1}^s\delta_i(\sum_{j=1}^{n_i}a_{i,j}\xx^{k_{i,j}})^2$, where $\delta_i,a_{i,j}\in \RR, k_{i,j}\in \NN^{|\xx|}$.  We say SOS decomposition $f=\sum_{i=1}^s\delta_i(\sum_{j=1}^{n_i}a_{i,j}\xx^{k_{i,j}})^2$ is more beautiful than decomposition $f=\sum_{i=1}^{s'}\delta'_i(\sum_{j=1}^{n'_i}a'_{i,j}\xx^{k'_{i,j}})^2$ if $\pi_1\left({\{\delta_i,a_{i,j} \}}\right)= \pi_1\left({\{\delta'_i,a'_{i,j} \}}\right) $ and $|\{\delta_i,a_{i,j} \}|\le |\{\delta'_i,a'_{i,j} \} | $.
%	
%\end{definition}
%
%\begin{example}
%Let $f=x^2+4x+7$, then SOS 	decomposition $f=(x+2)^2+3$ is  more beautiful than decomposition
%$f=\frac{1}{2}(x+3)^2+\frac{1}{2}(x+1)^2+2$.
%
%\end{example}
%		



\begin{theorem}
	Let 
\[\AA = 
\begin{pmatrix}
	a_{1,1} & a_{1,2} & \cdots & a_{1,n}& b_1 \\
	a_{2,1} & a_{2,2} & \cdots & a_{2,n}& b_2 \\
	\vdots  & \vdots  & \ddots & \vdots & \vdots \\
	a_{n,1} & a_{n,2} & \cdots & a_{n,n}& b_n  \\
	b_1     &  b_2    &  \cdots&   b_n  & c
\end{pmatrix}\]
where $a_{i,j} \in \RR, a_{i,j}=a_{j,i}, b_i\in \RR, c\in \RR, i,j=1,\cdots, n $. Let \[f=[\xx_1,\cdots, \xx_n,1]\AA[\xx_1,\cdots, \xx_n, 1]^T.\] 
Let $g=\sum_{i=1}^n\xx_i-C$. 
Let \[
B=\max\{a_{i,j}\frac{A^2}{2}  \mid i> j,\ i,j=1,\cdots,n \}, C=\min\{a_{i,j}\frac{A^2}{2}  \mid i> j,\ i,j=1,\cdots,n \}.\]  
If there exists $i,j,k\in \{1,\cdots, n\}$ such that
one of 1,2,3,4  in the following holds, then 
\begin{enumerate}
		\item  $a_{i,h}>a_{j,h}>a_{k,h}$ for $h=1,\cdots, n$, $b_{i}>b_{j}>b_{k}$ and $(d_j-d_i)(d_k-d_j)\le 0$
		\item  $a_{i,h}<a_{j,h}<a_{k,h}$ for $h=1,\cdots, n$, $b_{i}<b_{j}<b_{k}$ and $(d_j-d_i)(d_k-d_j)\le 0$
		\item  $a_{i,h}\ge a_{j,h}\ge a_{k,h}$ for $h=1,\cdots, n$, $b_{i}\ge b_{j}\ge b_{k}$ and $(d_j-d_i)(d_k-d_j)< 0$
		\item  $a_{i,h}\le a_{j,h}\le a_{k,h}$ for $h=1,\cdots, n$, $b_{i}\le b_{j}\le b_{k}$ and $(d_j-d_i)(d_k-d_j)< 0$
\end{enumerate}

.
	\begin{align}
&f \label{eq:obj4}\\
\mbox{s.\ t.}\quad &g=0 \nonumber\\
& x_i\ge 0,\quad i=1\cdots,n \nonumber
\end{align}

  $a_1,\cdots,a_n>0,$ $b_{i,j}\in \RR, i> j, i,j =1,\cdots, n$
	and $A>0$ are given constants.   Let \[
	 B=\max\{b_{i,j}\frac{A^2}{4a_{i}a_{j}}  \mid i> j,\ i,j=1,\cdots,n \}, C=\min\{b_{i,j}\frac{A^2}{4a_{i}a_{j}}  \mid i> j,\ i,j=1,\cdots,n \}.\]  
	 Let $f=\sum_{i=1}^n\sum_{j=1}^{i-1}b_{i,j}x_{i}x_{j}$.
	 Then $ \max\{0,B\}, \min\{0,C\}$ are the maximum, minimum values of (\ref{eq:obj}), respectively.
	 	\begin{align}
	 	&\sum_{i=1}^n\sum_{j=1}^{i-1}b_{i,j}x_{i}x_{j}  \label{eq:obj}\\
	 	\mbox{s.\ t.}\qquad	&\sum_{i=1}^na_ix_i=A \nonumber\\
	 	& x_i\ge 0,\quad i=1\cdots,n \nonumber
	 	\end{align}
%	 	
% Let \[D=\min\{b_{i,i}\left(\frac{A}{a_{i}}\right)^2\mid i=1,\cdots, n\},
% \ E=\min\{b_{i,j}\frac{A^2}{4a_{i}a_{j}}  \mid i> j,\ i,j=1,\cdots,n \}.\]  
%Then $ \min\{D,E\}$ is the minimum value of (\ref{eq:obj}).
\end{theorem}
\begin{proof}
	Let $g=\sum_{i=1}^na_ix_i-A$.
	We prove it by induction. When $n=2$, it is easy to check that the conclusion holds. Assume the conclusion holds
	when $n\le N$.
	
	If $n=N+1$.   In the following, we will prove that $f$ can achieve   $B,C$ in feasible area. Firstly, if $B=b_{k,k}\left(\frac{A}{a_{k}}\right)^2$, then let $\alpha$ is a vector such that 
	\[\alpha[i]=\begin{cases}
	\frac{A}{a_k}       & \quad   \text{ if  } i=k\\
	0  & \quad \text{ otherwise.}
	\end{cases} 	\]	
	 Then $\alpha$ satisfying (\ref{eq:obj})'s constraints and $f(\alpha)=B$. Secondly, if $C=b_{k,h}\frac{A^2}{4a_{k}a_{h}}, k\neq h$,
	 then let $\beta$  is a vector such that 
	 	\[\beta[i]=\begin{cases}
	 	\frac{A}{2a_k}       & \quad   \text{ if  } i=k\\
	 	\frac{A}{2a_h}       & \quad   \text{ if  } i=h\\
	 	0  & \quad \text{ otherwise.}
	 	\end{cases} 
	 	\]	
	  Then $\beta$ satisfying (\ref{eq:obj})'s constraints and $f(\beta)=C$.	
	
	  Let $\ksi=(\ksi_1,\cdots, \ksi_n)$ is a vector which makes $f$ to  achieve an extreme  value under constrains.  If there is $\ksi_i=0$, 
	  without loss of generality let $\ksi_1=0$, by the induction, the maximum value of (\ref{eq:obj2}) is $\max\{ B_1, C_1\}$
	  where 
	   \[\ B_1=\max\{b_{i,j}\frac{A^2}{4a_{i}a_{j}}  \mid i> j, i,j=2,\cdots,n \},
	   \ C_1=\min\{b_{i,j}\frac{A^2}{4a_{i}a_{j}}  \mid i> j, i,j=2,\cdots,n \}.\] 
	   It is easy to see that $B_1\le B$ and $C_1\ge C$. Additionally $g$ can  achieve $B,C$ in feasible area, hence $\max\{ B_1, 0\}=\max\{ B, 0\} $ and $\min\{ C_1, 0\}=\min\{ C, 0\} $, the conclusion holds.  Otherwise, $\ksi_i>0$ for $i=1,\cdots, n$, in other words $\ksi$ is an interior point of feasible area. By  KKT condition, there is a $\mu\in \RR$ such that  
	   \[\triangledown f(\ksi) = \mu\triangledown g(\ksi)  \]
	   where \[\triangledown f(\ksi)=[\frac{\partial f(\ksi)}{\partial x_1},\cdots, \frac{\partial f(\ksi)}{\partial x_n}], \ \triangledown g(\ksi)=[a_1,\cdots, a_n].\]
	   $\frac{\partial f(\ksi)}{\partial x_i}=\sum_{j=1}^{i-1}b_{i,j}\ksi_j+b_{i,i}\ksi_i=\mu a_i$.
	  	\begin{align}
	  	&\sum_{i=2}^n\sum_{j=2}^nb_{i,j}x_{i}x_{j}  \label{eq:obj2}\\
	  	\mbox{s.\ t.}\qquad	&\sum_{i=2}^na_ix_i=A \nonumber\\
	  	& x_i\ge 0,\quad i=2\cdots,n \nonumber
	  	\end{align}
\qed	
\end{proof}


\begin{problem}
	How to compute the minimum value of (\ref{eq:obj})?
\end{problem}
\begin{problem}
	  $a_1,\cdots,a_n\neq 0,l_i\cdots,l_n\in \RR, u_1,\cdots,u_n\in \RR, u_i\ge l_i$,  $b_{i,j}\in \RR, i,j=1,\cdots, n$ and $A$ are given constants. 
	\begin{align}
	&\sum_{i=1}^n\sum_{j=1}^nb_{i,j}x_{i}x_{j}  \label{opt:cost}\\
	\mbox{s.\ t.}\qquad	&\sum_{i=1}^na_ix_i=A \nonumber\\
	&l_i\le x_i\le u_i\quad i=1,\cdots,n \nonumber
	\end{align}
	where $x_i\in \RR,\ i =1,\cdots, n$.
How to compute the maximum and minimum values of (\ref{opt:cost})?
\end{problem}
	


\end{document}